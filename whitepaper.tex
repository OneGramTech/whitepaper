\documentclass[letterpaper,11pt]{article}

\usepackage{graphicx}
%\usepackage{fullpage}
\usepackage{pdfpages}
\usepackage{color}
\usepackage[colorlinks=true,urlcolor=blue,citecolor=black]{hyperref}
\usepackage{url}
\usepackage[font=footnotesize,labelfont=bf]{caption}
%full name for appendix
\usepackage[title]{appendix}
\usepackage{float}
%\usepackage{parskip}
%for code
\usepackage{listings}
%for math
\usepackage{amsmath}

\linespread{1.1}
\setlength{\parindent}{30pt}
\setlength{\emergencystretch}{3em}


%opening
\title{\LARGE OneGram:\\
	\Large A Shariah Compliant and Gold Backed Digital Token}
\author{
		I.M. Khan
	}
\date{May 3, 2017\\\small Version 1.0.0}
\begin{document}

\maketitle

\begin{abstract}

About eight years ago, a pseudonymous cryptographer known as Satoshi Nakamoto introduced Bitcoin as a digital analog to gold: Limited in supply, but secured by modern cryptography, and made for the internet age.
Following Satoshi’s footsteps, many tried to improve on Satoshi’s original vision, and thousands of alternative cryptocurrencies were born.

Despite significant recent developments and innovation, the market for cryptocurrencies remains very niche. Cryptocurrencies still have shortcomings that discourage mainstream use, in particular high volatility and barriers to entry. It’s also worth noting that existing cryptocurrencies have not been designed with Islamic markets in mind. While the 1.6 billion Muslims make up over 23 percent of the world population, many Muslims simply can’t use cryptocurrencies because of their restricted legal status and high barriers to entry in many countries in the Islamic world.

OneGram aims to solve these issues by using blockchain technology to create a new kind of cryptocurrency, where each coin is backed by one gram of gold at launch.

In addition, each transaction of OneGram Coin (OGC) generates a small transaction fee which is reinvested in more gold (net of admin costs), thus increasing the amount of gold that backs each OneGram. Therefore, each OGC increases in real value over time, making OneGram unique among cryptocurrencies.

\end{abstract}

\section{Cryptocurrencies}

Bitcoin and alternative cryptocurrencies are starting to see widespread adoption. The main innovation behind cryptocurrencies is that, instead of relying on a trusted third party, transactions are recorded and propagated in a distributed ledger known as “blockchain.” This allows transactions to be trustless, censorship-resistant, permissionless and private. Once a transaction is confirmed by the blockchain network, it becomes irreversible: it can’t be charged back through a dispute process like other forms of money transfer.

Cryptocurrencies promise to radically change how we do banking by removing artificial barriers caused by legacy financial institutions, allowing for:

\begin{itemize}
	\item True peer-to-peer payments anywhere in the world.
	\item Minimal transaction fees and processing time compared to traditional banking.
	\item Payments between pseudonymous parties ensuring financial privacy.
	\item Non-reversible transactions preventing chargebacks and fraud.
\end{itemize}

\section{Sharia and Islamic Finance}

Sharia is a set of religious laws derived from the Islamic tradition, in particular, the Quran and Hadith. It includes principles that guide the financial, economic, legal, and domestic affairs of Muslims.

Islamic finance prohibits unjustified enrichment, as well as engaging in transactions with excessive risk or speculation (such as gambling), in order to establish justice and avoid exploitation in business transactions.
Accordingly, Islamic scholars derived 3 key criteria that differentiate Islamic finance from its counterparts:


\begin{itemize}
	\item Prohibiting interest, which is the source of unjustifiable income.
	\item Encouraging profit and loss sharing based on the partners’ capital, risks share and effort.
	\item Prohibiting transactions that are uncertain and excessively speculative.
\end{itemize}

\subsection{Money According to Sharia}

Most Muslims today have no idea that the money they use is arguably not Sharia-compliant. Most of the world uses fiat currency, which is money backed only by legal tender laws. Historically, money was either created from or backed by precious metals. That said, what is money in Islam?

One of the most reliable Islamic scholars on the subject is Imran Hosein\cite{hosein}, who specializes in modern socio-economic and political issues. He explains the six key properties of money in Islam as:

\begin{enumerate}
	\item Money is either precious metals or food.
	\item Money is abundant and widely available.
	\item Money is durable and does not spoil or corrode.
	\item Money has intrinsic value.
	\item Money exists in creation and is made valuable by God.
	\item Money functions as a medium of exchange.
\end{enumerate}

Fiat money would comply with the first point if backed by metal or food. Unfortunately, countries no longer back their currencies with commodities anymore. This has been the case since the U.S. canceled the convertibility of the United States dollar to gold with the “Nixon Shock” in the 70s\cite{nixon}. Since then, national currencies have had freely floating exchange rates, dictated mainly by central banks’ monetary policies\cite{jeffrey}.

Fiat currencies are indeed abundant and widely available within the jurisdictions where they are accepted.

Modern paper money and coins, along with their digital counterpart, are durable enough as required by the third point.

Does fiat money have intrinsic value? Since the Nixon Shock of 1971, the exchange rates of national currencies are free floating, so fiat money can no longer claim to have an intrinsic value. It is backed by the government issuing it, but that isn’t intrinsic value.

Fiat currency is made valuable by a government’s artificial monetary policy, so clearly the fifth point is not respected.

As for the last point, fiat money is indeed used as a medium of exchange. In fact, one can claim that this is the only use for fiat money.

So, fiat money complies with, at best, three of the six properties of money in Islamic law. What about cryptocurrencies like Bitcoin?

While Bitcoin and many other cryptocurrencies are much closer to the Islamic definition of money than modern fiat money, they still fall short, as they are not backed by any tangible real world asset. So there is a need for a better option.

Each OneGram is backed by a minimum of one gram of gold. OneGram is abundant. When the coins will go on sale at the Initial Coin Offering (ICO), there will be 12,400,786 tokens. All of them will be available for purchase through our partner gold exchange, GoldGuard, an online gold trading platform that enables customers to buy and sell gold at spot rates and physically store it. Following the ICO, anyone will be able to buy and sell OneGram freely through many international channels.

OGCs are durable and can be kept safely in your cryptographically secure wallet. OneGram has an intrinsic value because it’s backed by physical gold. 

The OneGram ecosystem will offer merchant apps and payment processing services, making OneGram an excellent medium of exchange all around the world.

OneGram is a comprehensive Sharia compliant product and follows the three basic criteria for trading in Islamic finance: There is no interest mechanism in the issuance of OneGram, profit-loss sharing is part of the rewards scheme, and there is minimal speculation as OneGram is an asset backed by physical gold.


\subsection{Gold Investments and Sharia}

Historically, gold-investing has been problematic under Sharia law. While there is currently limited guidance for gold coins and bars, there is virtually no guidance on gold elsewhere in the financial sector\cite{bakar}.

For example, in most cases, trading gold futures contracts is forbidden by Islamic law, because gold futures contracts aren’t backed by physical gold and you can end up paying or receiving interest on your trading account. As a result, most people who wanted to buy gold as an investment purchased gold in its physical form, such as jewelry or coins.

In December 2016, the Sharia Gold Standard was introduced by the Accounting and Auditing Organization for Islamic Financial Institutions (AAOIFI), the World Gold Council and Amanie Advisors. With this new standard, Muslim investors will now be able to take advantage of an increasing range of gold-backed investment opportunities, which had previously been non-compliant.

OneGram complies with the Sharia Gold Standard. So you can feel safe knowing that your investment will respect your values as well as providing you with financial growth

\section{Limited Downside Unlimited Upside
}

The history of fiat currency is a history of volatility. The average lifespan of fiat currency is only 27 years old\cite{pento}. Even if a currency survives, invariably it will experience inflation. With central banks having the power to print as much currency as they please, combined with the destructive effects of inflation, the purchasing power of fiat money experiences a steady decline. The world’s oldest fiat currency, the British pound, is an excellent example: it has lost 99.5 percent of its value since inception.

Historically gold is more resilient, and holds its worth better than any fiat currency, particularly in times of economic instability. No currency can guarantee absolute stability, but OneGram limits your exposure to the downside risk. Since the base price of OneGram is always at least equal to the spot price of gold, OneGram has a floor price.

What’s more is that usage and market demand also adds a premium to the value of OneGram. Therefore, OneGram has a three-part valuation system to determine its market price.

The first part is the Gold Value (GV), with the value being determined by the spot price of gold. The second part is the present value of the transaction fees reinvested to buy more gold (TF), with the value being determined by the usage of OGC. The last part is the Demand Premium (DP), with the value being determined by market demand. This creates the following formula for the market price: 

\[ OneGram Value = GV + TF + DP  \]


\section{Growth With Every Transaction}

Each OneGram transaction generates a 1\% transaction fee, up to a maximum of 1 OGC. Unlike other cryptocurrencies, in OneGram, 70\% of this fee is reinvested to buy more gold and increase the amount of gold that backs each token. As transaction volume increases, more gold gets added to the vault and all OneGram owners share in the profit. So, over time, the value of each OneGram rises by design. This makes OneGram a unique asset whose value increases perpetually. 25\% of this fee will be used for development and Operations. 2.5\% will be donated to Charities and 2.5\% will reward miners (POS staking).

The transaction volume for a new cryptocurrency is typically low. To address this, we will be introducing a variety of tools for rapid real-world adoption of OneGram as a currency.

OneGram Coins are issued and redeemed for gold via the the GoldGuard platform. Following the ICO, the OGCs can be bought, sold and traded via any major cryptocurrency trading platform.

A new payment gateway, YalaPay, will be launched for OneGram Coins. Besides fiat conversion, YalaPay will include marketing tools such as a white label loyalty program for merchants, featuring special offers, and discounts for customers. The model will be first introduced in Dubai and Abu Dhabi, with the payment institution license already in place.

A GoldGuard Mastercard debit card (“Liquid Gold”) will also be created, which will work across the globe in ATMs, POS systems and online. It will be possible to recharge the Liquid Gold card with Fiat, OGC or Gold through our payment gateway, online or through GoldGuard ATM machines.

By giving customers all the best financial services offered by cryptocurrencies and more, OneGram will make all transactions easy for the users.

\section{Technical Specifications}

OneGramCoin uses a unique proof-of-stake blockchain with over 3 months of development by expert cryptocurrency engineers. Our fully-customized blockchain takes inspiration from Bitcoin, Dash and BlackCoin. All transactions are near-instant and private. We strengthened the consensus mechanism by introducing trusted nodes operated by verified entities. The cumulative 2.5\% transaction fees distributed in a block reward to the trusted nodes will maintain a secure blockchain. The OneGram blockchain block size limit is 1MB with an average 1 minute block time.

\section{Smooth Entry and Exit With GoldGuard}

Buying and selling cryptocurrencies like Bitcoin is not always an easy task. Depending on where you live, there may be no obvious entry and exit point. Mining cryptocurrencies is even more difficult, requires costly equipment, and is definitely not for the average user. This is one of the reasons we are doing our ICO with GoldGuard, offering the investors an easy way to acquire OneGram Coins.

Once a OneGram Coin is issued, it can be redeemed for physical gold from GoldGuard at any time. Any OneGram Coin redeemed for gold or equivalent in fiat currency is sent to a publicly verifiable burn address, permanently removing the coin from circulation.

GoldGuard is licensed with Dubai Airport Free Zone (DAFZ) to trade jewelry, namely gold. Established in 1996, within the boundaries of Dubai International Airport, DAFZ is one of the fastest growing Free Zones in the region.

All gold is safeguarded by our logistics and transportation expert Loomis. All accounts at GoldGuard are asset-backed, audited by PwC, and insured against theft and damage.


\section{Giving Back to the Community}

Giving back to the community and setting a good example for others is central to how we do business. We seek to contribute to social and economic progress both globally and locally.

With that in mind, we created the OneGram Foundation (OGF). OGF will take 2.5\% of the total transaction fees that OneGram products generate and donate it to local and international charities.

By sharing our success, we aim to bring relief to the lives of the less fortunate.

OGF supports the principles of Islamic sharing and caring. The OneGram community is contributing, through corporate social responsibility, to humanitarian causes.

\section{OneGram ICO, How You Can Participate}

The OneGram ICO will take place on the GoldGuard gold exchange.

The first step to participate in the ICO is to register with GoldGuard and purchase gold. Then the gold can be redeemed for OGC via the same platform. There is a fee of 10\% charged during the purchase; this fee is expected as by purchasing OGC, the investor is not only purchasing an asset that offers the spot value of gold but also the future value of additional gold to be purchased from transaction fees.

The maximum supply of OGC is 12,400,786. The ICO starts on May 21st, 2017, and will end when all coins are sold or after a maximum of 120 days. If the tokens do not sell out, there will be a new total supply of OGC equal to the amount of OGC sold in the ICO. After that, no more coins will ever be issued.

At any given moment, you can see the amount of gold backing your coin in the official OneGram wallet app and the GoldGuard website. Through GoldGuard, you are able to redeem your coins for gold or equivalent fiat currency. After the ICO, you will also be able to buy and sell OneGram Coins through any cryptocurrency exchange that lists OGC.

%bibliography
\bibliographystyle{unsrt}
\bibliography{whitepaper}

\end{document}

